%intended compiler: LuaLaTeX
\documentclass[12pt]{article}

\usepackage[utf8]{inputenc}
\usepackage[T1]{fontenc}
\usepackage{amsmath, amssymb, tikz, lmodern}
\usetikzlibrary{graphs, graphdrawing, shapes.geometric}
\usegdlibrary{force}
\usepackage[spanish]{babel}
\usepackage{geometry}
\newgeometry{vmargin= {15mm}, hmargin= {12mm, 17mm}}

\setlength{\parindent}{0pt}

\title{Notas de Matemática Discreta II}
\date{2025}
\author{Ignacio Gomez Barrios}

\begin{document}
\newpage

\begin{center}
  \Huge \textbf{Notas de Matemática Discreta II} \\[3cm]
  \Large 2025 \\[4cm]
  \Large \textit{Autor: Ignacio Gomez Barrios}
\end{center}

\newpage

\section*{Clase del 12/3}
\subsection*{Grafos}
\underline{\textbf{Definición:}} 

Un grafo es un par \((V, E)\) donde \(V\) es un conjunto y \(E\) es un \textbf{subconjunto} del conjunto de 2 elementos de \(V\)

\(E \subseteq \left\{A \subseteq V : |A| = 2\right\}\) \bigskip

\underline{\textbf{Notación:}}

\(\left\{x, y\right\}\) se denotará \(xy \therefore xy = yx\)

Ejemplo de grafo aqui: \bigskip

 El \textbf{vecindario} de un vertice \(x\) es :

\(\Gamma(x) = \left\{y \in V : xy \in E\right\}\) \\

El \textbf{grado} de un vertice \(x\) es:

\(d(x) = |\Gamma(x)|\) \\

\(\delta = min\left\{d(x) : x \in V\right\}\)

\(\Delta = max\left\{d(x) : x \in V\right\}\) \\

Un grafo se dice regular si \(\delta = \Delta\) \bigskip

Ejemplos de grafos "famosos"

\begin{enumerate}
\item 

\(C_n\) son los ciclos con \(n\) elementos, con \(n \geq 3\)

\(V = \left\{1, 2, 3, ..., n\right\}\)

\(E = \left\{12, 23, ..., (n-1)n, n1\right\}\)

ejemplo de \(C_n\) aqui

\item
\(K_n = \) grafo completo en \(n\) vertices, con \(n \in \mathbb{N}\) 

\(V = \left\{1, 2, ..., n\right\}\)

\(E = \left\{A \subseteq V : |A| = 2\right\}\)

ejemplo de \(K_n\) aqui

\item
\(K_{r, t} \) es el grafo \(V = \left\{X \cup Y\right\}\), \(E = \left\{xy : x \in X, y \in Y\right\}\) 

ejemplo de \(K_{r, t}\) aqui

Este el \textit{"grafo bipartito completo"}
\end{enumerate}
\bigskip

\underline{\textbf{Definición:}}

un \textbf{camino} entre \(x\) e \(y\) es una sucesión de vertices \(z_0, z_1, ..., z_t\) tales que \(z_0 = x\), \(z_t = y\) y \(\forall i  : 0 \leq i \leq t\) tenemos  \(z_{i}z_{i+1} \in E\)

Entonces denotamos 

\(x \sim y \iff \exists\) un camino desde \(x\) a \(y\)

$\sim$ es una relación de equivalencia $\rightarrow$ induce una partición de $V$ en clases de equivalencia, las cuales se llaman componentes conexas. \bigskip

\underline{\textbf{Definición:}}

Un Grafo es conexo si tiene 1 sola componente conexa

\subsection*{Coloreo de grafos}

Un coloreo de un grafo $G = (V, E)$ con $k$ colores es una función 

\(C : V \rightarrow \left\{1, ..., k\right\}\) tal que \(xy \in E \implies C(x) \neq C(y)\) \bigskip

\underline{\textbf{Definición:}}

El número cromático de un grafo $G$ es

\(\chi(G) = min\left\{k : \exists \text{ un coloreo con } k \text{ colores}\right\} \) \\

\underline{\textbf{Problema:}}

calcular $\chi(G)$ eficientemente

Algoritmo de fuerza bruta: colorear cada vertice con algun color $1, ..., k$ y chequear si es propio

(complejidad $O(n^n)$) \bigskip

\subsubsection*{Algoritmo de Greedy para coloreo}

requiere un \textbf{orden} de vertices 

\(x_1, x_2, ..., x_n\) en algun orden \bigskip

\(
Greedy 
\begin{cases}
C(x_1) = 1 \\
C(x_i) = min\left\{k : k \text{ no sea color de ningun } x_1, x_2, ..., x_{i - 1} \text{ que sea vecino de } x_i\right\}
\end{cases}
\) \\

invariante de $Greedy$: los coloreos son propios \bigskip

$Greedy$ no calcula necesariamente $\chi(G)$

por ejemplo, en $C_6$ con el orden $1, 4, 2, 3, 5, 6$

\[
\begin{array}{|c|}
\hline
\textbf{1} \\
\hline
1 \\
\hline
\end{array}
\rightarrow
\begin{array}{|c|}
\hline
\textbf{1} \\
\hline
1 \\
\hline
4 \\
\hline
\end{array}
\rightarrow
\begin{array}{|c|c|}
\hline
\textbf{1} &  \textbf{2}\\
\hline
1  & 2 \\
\hline
4 \\
\hline
\end{array}
\rightarrow
\begin{array}{|c|c|c|}
\hline
\textbf{1} &  \textbf{2} & \textbf{3} \\
\hline
1  & 2  & 3 \\
\hline
4 \\
\hline
\end{array}
\rightarrow
\begin{array}{|c|c|c|}
\hline
\textbf{1} &  \textbf{2} & \textbf{3} \\
\hline
1  & 2  & 3 \\
\hline
4  & 5 \\
\hline
\end{array}
\rightarrow
\begin{array}{|c|c|c|}
\hline
\textbf{1} &  \textbf{2} & \textbf{3} \\
\hline
1  & 2  & 3 \\
\hline
4  & 5  & 6\\
\hline
\end{array}
\]

\(
\implies Greedy \text{ da un coloreo con } 3 \text{ colores, pero } 
\begin{cases}
C(x_1) = 1 & i \text{ impar}\\
C(x_i) = 2 & i \text{ par}
\end{cases} \text{ es propio} 
\) 

\(
\implies Greedy(G) \neq \chi(G) = 2
\) \bigskip

una mirada a los $C_n$

similar al ejemplo, si $n$ es par 
\(\implies C(i) = \begin{cases} 1 & i \text{ impar} \\ 2 & i \text{ par}\end{cases}\)

\subsubsection*{Propiedad:}

si $n$ es impar $\implies$ $\chi(C_n) = 3$ \bigskip

\underline{\textbf{Demostración:}}

veamos primero que $\chi(C_n) \geq 3$ si $n$ es impar \bigskip

supongamos que no $\implies \exists$ un coloreo propio con 2 colores \bigskip

sea $A$ el color del vertice 1

$12 \in E \implies C(2) \neq C(1) = A$, como solo hay 2 colores,  $C(2) = B$

$23 \in E \implies C(3) \neq C(2) = B$, como solo hay 2 colores $C(3) = A$ \bigskip

generalizando, tenemos $C(i) = A$ si $i$ es impar, $C(i) = B$ si $i$ es par \bigskip

$\implies C(n) = A$ (pues $n$ impar) y $C(1) = A$, pero $n1 \in E$
$\implies C(n) \neq C(1)$ \textbf{Absurdo} \bigskip

veamos ahora que $\chi(C_n) \leq 3$


\[
\text{sea el coloreo: }
C(i) =
\begin{cases}
1 & i \text{ impar, } i < n \\
2 & i \text{ par} \\
3 & i = n
\end{cases}
\]

es propio $\therefore \chi(C_n) \leq 3$ \bigskip

\textbf{observación parcial:}

sea $H = (W, F)$ un \textit{subgrafo} de $G = (V, E)$ 

(es decir, $H$ es un grafo y $W \subseteq V, F \subseteq E$)

\(\implies \chi(H) \leq \chi(G)\) \bigskip

\underline{\textbf{Corolario:}}

si $G$ tiene algún ciclo impar como subgrafo, entonces $\chi(G) \geq 3$ \bigskip

\underline{\textbf{Observaciones:}}

\begin{itemize}
\item $\chi(G) \leq n$ con $n =$ número de vertices
\item $\chi(K_n) = n$
\item Si $K_r$ es un subgrafo de G $\implies r \leq \chi(G)$
\end{itemize}

\newpage

\section*{Clase del 14/3}
Lorem ipsum dolor sit amet, consectetur adipiscing elit, sed do eiusmod tempor incididunt ut labore et dolore magna aliqua. Ut enim ad minim veniam, quis nostrud exercitation ullamco laboris nisi ut aliquip ex ea commodo consequat. Duis aute irure dolor in reprehenderit in voluptate velit esse cillum dolore eu fugiat nulla pariatur. Excepteur sint occaecat cupidatat non proident, sunt in culpa qui officia deserunt mollit anim id est laborum.
\newpage

\section*{Clase del 19/3}
Lorem ipsum dolor sit amet, consectetur adipiscing elit, sed do eiusmod tempor incididunt ut labore et dolore magna aliqua. Ut enim ad minim veniam, quis nostrud exercitation ullamco laboris nisi ut aliquip ex ea commodo consequat. Duis aute irure dolor in reprehenderit in voluptate velit esse cillum dolore eu fugiat nulla pariatur. Excepteur sint occaecat cupidatat non proident, sunt in culpa qui officia deserunt mollit anim id est laborum.
\newpage

\section*{Clase del 21/3}
Lorem ipsum dolor sit amet, consectetur adipiscing elit, sed do eiusmod tempor incididunt ut labore et dolore magna aliqua. Ut enim ad minim veniam, quis nostrud exercitation ullamco laboris nisi ut aliquip ex ea commodo consequat. Duis aute irure dolor in reprehenderit in voluptate velit esse cillum dolore eu fugiat nulla pariatur. Excepteur sint occaecat cupidatat non proident, sunt in culpa qui officia deserunt mollit anim id est laborum.

\end{document}
   
