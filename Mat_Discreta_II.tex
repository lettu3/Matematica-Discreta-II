%intended compiler: LuaLaTeX
\documentclass[12pt]{article}

\usepackage[utf8]{inputenc}
\usepackage[T1]{fontenc}
\usepackage{amsmath, amssymb, tikz, lmodern}
\usetikzlibrary{graphs, graphdrawing, shapes.geometric}
\usegdlibrary{force}
\usepackage[spanish]{babel}
\usepackage{geometry}
\newgeometry{vmargin= {15mm}, hmargin= {12mm, 17mm}}

\setlength{\parindent}{0pt}

\title{Notas de Matemática Discreta II}
\date{2025}
\author{Ignacio Gomez Barrios}

\begin{document}
\newpage

\begin{center}
  \Huge \textbf{Notas de Matemática Discreta II} \\[3cm]
  \Large 2025 \\[4cm]
  \Large \textit{Autor: Ignacio Gomez Barrios}
\end{center}

\newpage

\section*{Clase del 12/3}
\subsection*{Grafos}
\underline{\textbf{Definición:}} 

Un grafo es un par \((V, E)\) donde \(V\) es un conjunto y \(E\) es un \textbf{subconjunto} del conjunto de 2 elementos de \(V\)

\(E \subseteq \left\{A \subseteq V : |A| = 2\right\}\) \bigskip

\underline{\textbf{Notación:}}

\(\left\{x, y\right\}\) se denotará \(xy \therefore xy = yx\)

Ejemplo de grafo aqui: \bigskip

 El \textbf{vecindario} de un vertice \(x\) es :

\(\Gamma(x) = \left\{y \in V : xy \in E\right\}\) \\

El \textbf{grado} de un vertice \(x\) es:

\(d(x) = |\Gamma(x)|\) \\

\(\delta = min\left\{d(x) : x \in V\right\}\)

\(\Delta = max\left\{d(x) : x \in V\right\}\) \\

Un grafo se dice regular si \(\delta = \Delta\) \bigskip

Ejemplos de grafos "famosos"

\begin{enumerate}
\item 

\(C_n\) son los ciclos con \(n\) elementos, con \(n \geq 3\)

\(V = \left\{1, 2, 3, ..., n\right\}\)

\(E = \left\{12, 23, ..., (n-1)n, n1\right\}\)

ejemplo de \(C_n\) aqui

\item
\(K_n = \) grafo completo en \(n\) vertices, con \(n \in \mathbb{N}\) 

\(V = \left\{1, 2, ..., n\right\}\)

\(E = \left\{A \subseteq V : |A| = 2\right\}\)

ejemplo de \(K_n\) aqui

\item
\(K_{r, t} \) es el grafo \(V = \left\{X \cup Y\right\}\), \(E = \left\{xy : x \in X, y \in Y\right\}\) 

ejemplo de \(K_{r, t}\) aqui

Este el \textit{"grafo bipartito completo"}
\end{enumerate}
\bigskip

\underline{\textbf{Definición:}}

un \textbf{camino} entre \(x\) e \(y\) es una sucesión de vertices \(z_0, z_1, ..., z_t\) tales que \(z_0 = x\), \(z_t = y\) y \(\forall i  : 0 \leq i \leq t\) tenemos  \(z_{i}z_{i+1} \in E\)

Entonces denotamos 

\(x \sim y \iff \exists\) un camino desde \(x\) a \(y\)

$\sim$ es una relación de equivalencia $\rightarrow$ induce una partición de $V$ en clases de equivalencia, las cuales se llaman componentes conexas. \bigskip

\underline{\textbf{Definición:}}

Un Grafo es conexo si tiene 1 sola componente conexa

\subsection*{Coloreo de grafos}

Un coloreo de un grafo $G = (V, E)$ con $k$ colores es una función 

\(C : V \rightarrow \left\{1, ..., k\right\}\) tal que \(xy \in E \implies C(x) \neq C(y)\) \bigskip

\underline{\textbf{Definición:}}

El número cromático de un grafo $G$ es

\(\chi(G) = min\left\{k : \exists \text{ un coloreo con } k \text{ colores}\right\} \) \\

\underline{\textbf{Problema:}}

calcular $\chi(G)$ eficientemente

Algoritmo de fuerza bruta: colorear cada vertice con algun color $1, ..., k$ y chequear si es propio

(complejidad $O(n^n)$) \bigskip

\subsubsection*{Algoritmo de Greedy para coloreo}

requiere un \textbf{orden} de vertices 

\(x_1, x_2, ..., x_n\) en algun orden \bigskip

\(
Greedy 
\begin{cases}
C(x_1) = 1 \\
C(x_i) = min\left\{k : k \text{ no sea color de ningun } x_1, x_2, ..., x_{i - 1} \text{ que sea vecino de } x_i\right\}
\end{cases}
\) \\

invariante de $Greedy$: los coloreos son propios \bigskip

$Greedy$ no calcula necesariamente $\chi(G)$

por ejemplo, en $C_6$ con el orden $1, 4, 2, 3, 5, 6$

\[
\begin{array}{|c|}
\hline
\textbf{1} \\
\hline
1 \\
\hline
\end{array}
\rightarrow
\begin{array}{|c|}
\hline
\textbf{1} \\
\hline
1 \\
\hline
4 \\
\hline
\end{array}
\rightarrow
\begin{array}{|c|c|}
\hline
\textbf{1} &  \textbf{2}\\
\hline
1  & 2 \\
\hline
4 \\
\hline
\end{array}
\rightarrow
\begin{array}{|c|c|c|}
\hline
\textbf{1} &  \textbf{2} & \textbf{3} \\
\hline
1  & 2  & 3 \\
\hline
4 \\
\hline
\end{array}
\rightarrow
\begin{array}{|c|c|c|}
\hline
\textbf{1} &  \textbf{2} & \textbf{3} \\
\hline
1  & 2  & 3 \\
\hline
4  & 5 \\
\hline
\end{array}
\rightarrow
\begin{array}{|c|c|c|}
\hline
\textbf{1} &  \textbf{2} & \textbf{3} \\
\hline
1  & 2  & 3 \\
\hline
4  & 5  & 6\\
\hline
\end{array}
\]

\(
\implies Greedy \text{ da un coloreo con } 3 \text{ colores, pero } 
\begin{cases}
C(x_1) = 1 & i \text{ impar}\\
C(x_i) = 2 & i \text{ par}
\end{cases} \text{ es propio} 
\) 

\(
\implies Greedy(G) \neq \chi(G) = 2
\) \bigskip

una mirada a los $C_n$

similar al ejemplo, si $n$ es par 
\(\implies C(i) = \begin{cases} 1 & i \text{ impar} \\ 2 & i \text{ par}\end{cases}\)

\subsubsection*{Propiedad:}

si $n$ es impar $\implies$ $\chi(C_n) = 3$ \bigskip

\underline{\textbf{Demostración:}}

veamos primero que $\chi(C_n) \geq 3$ si $n$ es impar \bigskip

supongamos que no $\implies \exists$ un coloreo propio con 2 colores \bigskip

sea $A$ el color del vertice 1

$12 \in E \implies C(2) \neq C(1) = A$, como solo hay 2 colores,  $C(2) = B$

$23 \in E \implies C(3) \neq C(2) = B$, como solo hay 2 colores $C(3) = A$ \bigskip

generalizando, tenemos $C(i) = A$ si $i$ es impar, $C(i) = B$ si $i$ es par \bigskip

$\implies C(n) = A$ (pues $n$ impar) y $C(1) = A$, pero $n1 \in E$
$\implies C(n) \neq C(1)$ \textbf{Absurdo} \bigskip

veamos ahora que $\chi(C_n) \leq 3$


\[
\text{sea el coloreo: }
C(i) =
\begin{cases}
1 & i \text{ impar, } i < n \\
2 & i \text{ par} \\
3 & i = n
\end{cases}
\]

es propio $\therefore \chi(C_n) \leq 3$ \bigskip

\textbf{observación parcial:}

sea $H = (W, F)$ un \textit{subgrafo} de $G = (V, E)$ 

(es decir, $H$ es un grafo y $W \subseteq V, F \subseteq E$)

\(\implies \chi(H) \leq \chi(G)\) \bigskip

\underline{\textbf{Corolario:}}

si $G$ tiene algún ciclo impar como subgrafo, entonces $\chi(G) \geq 3$ \bigskip

\underline{\textbf{Observaciones:}}

\begin{itemize}
\item $\chi(G) \leq n$ con $n =$ número de vertices
\item $\chi(K_n) = n$
\item Si $K_r$ es un subgrafo de G $\implies r \leq \chi(G)$
\end{itemize}

\newpage

\section*{Clase del 14/3}
\subsection*{BFS y DFS}

tanto $BFS(x)$ como $DFS(x)$ $\rightarrow$ encuentran la componente conexa de $x$

\begin{enumerate}
\item $C = \left\{x\right\}$
$T = \left\{x\right\}$ \hspace{1em} \textit{Spanning Tree} \\
$P, Q = [x]$ \hspace{1em} P = pila, Q = cola, $DFS$ usa una pila, $BFS$ usa una cola

\item 
\texttt{while} \\
\hspace{2em} $Q \neq \emptyset$ \\
\hspace{2em} $P \neq \emptyset$
\end{enumerate}

A partir de aqui, los algoritmos divergen \bigskip

\underline{\textbf{BFS}}

\begin{enumerate}
\item$\forall y \in \Gamma(z) \cap (V -C)$

$C = C \cup \left\{y\right\}$

$Q$ += $y$

$T$ += $y $

$E(T) = E(T) \cup \left\{y\right\}$

\item remover $z$ de $Q$

\item\texttt{return } $C, T$
\end{enumerate}

\underline{\textbf{DFS}}

\texttt{if \(\Gamma(z) \cap (V - C) = \emptyset\)}

\hspace{1em} borrar $z$ de $P$

\texttt{else }

\hspace{1em}tomar \textbf{algún} $y \in \Gamma(z) \cap (V-C)$ 

\hspace{1em}$C = C \cup \left\{y\right\}$, agregar $y$ a $P$, $T$

\hspace{1em}agregar $yz$ a $E(t)$

\texttt{return } $C, T$ \bigskip

\subsection*{Problema : "Familia de problemas"}

para cada $k \in \mathbb{N}$ definimos el problema $k-color$ 
como:\bigskip

dado grafo $G$, es $\chi(G) \leq k$? \bigskip

\begin{center}
\subsection*{Teorema}
$2-color$ es polinomial
\end{center}

\underline{\textbf{Demostracion:}}

daremos un algoritmo que resuelve el problema y es polinomial

observacion: \(\chi(G) \leq 2 \iff \chi(C) \leq 2,  \forall \text{ componente conexa de } G\)

vamos a suponer que G es conexo

\underline{\textbf{algoritmo:}}
\begin{enumerate}
\item tomar $x \in V(G)$
\item correr $BFS(x)$, sea $T(x)$ el $tree$
\item colorear los vertices de $V(G)$ como \(C(z) \begin{cases} 1 & \text{ si } nivel_{T(x)}(z)\text{ es impar} \\ 2 & \text{ si } nivel_{T(x)}(z) \text{ es par} \end{cases}\)
\item \texttt{if } (coloreo del paso 3 es propio)

\hspace{1em}resultado = "si"

\texttt{else}

\hspace{1em}resultado = "no"

\end{enumerate}\bigskip

Complejidad:

$BFS(x) \rightarrow$ \(O(n) \\ O(\sum d(z)) = O(2n) = O(n) \)\bigskip

el paso 3 es 'gratis' cuando un vertice $w$ agrega un vertice $v$

\(C(v) =
\begin{cases}
1 & \text{ si } C(w) = 2 \\
2 & \text{ si } C(w) = 1
\end{cases}
\)\bigskip

paso 4: $O(n)$ \bigskip

\underline{\textbf{Correctitud:}}

si devuelve ''si'' es porque el coloreo es propio $\implies \chi(G) \leq 2$\bigskip

veamos que pasa si devulve ''no''

para probar eso veremos que $G$ tiene un ciclo impar\bigskip

si la respuesta es ''no'' $\implies$ el coloreo del paso $|3|$ no es propio \(\implies zv \in E(G), C(z) = C(v)\)\bigskip

esto significa:

\(
C(z) =
\begin{cases}
1 & nivel_{T(x)}(z) \text{ par} \\
2 & nivel_{T(x)}(z)\text{ impar}
\end{cases}
\hspace{1em}C(v) = 
\begin{cases}
1 & nivel_{T(x)}(v) \text{ par} \\
2 & nivel_{T(x)}(v)\text{ impar}
\end{cases}
\)\bigskip

$\implies$ $nivel(z)$ y $nivel(v)$ son ambos impares $\vee$ ambos pares

\(\implies[nivel(z) + nivel(v) \text{ es par}]\circledast\)\bigskip

sea $xz_{1}z_{2} ... z_{j}$ con $z_j = z$ el único camino entre $x$ y $z$ en $T(x)$

\(\implies nivel(z) = j \)\bigskip

sea $xv_{1}v_{2}...v_i$ con $v_i = v$ el único camino entre $x$ y $v$ en $T(x)$

\(\implies nivel(v) = i\)\bigskip

esos caminos empiezan igual (en $x$) y terminan distinto (en $z$ y $v$) 

$\therefore \exists w$ tal que 

$xz_{1}z_{2}...z_{j} = xz_{1}z_{2}...wz_{p+1}...z_{j}$ 

$xv_{1}v_{2}...v_{i} = xv_{1}v_{2}...wv_{p+1}...v_{i}$

con $p = nivel(w)$  \bigskip

como $zv \in E(G)$, en $G$ tenemos el ciclo

$wz_{p+1}z_{p+2}...z_{j}v_{i}v_{i-1}...v_{p+1}w$\bigskip

tenemos el ciclo $1 + (j - p) + (i -p) = 1  + p(i + j) =$ impar

\(\implies \chi(G) \nleq 2\)\bigskip

\underline{\textbf{Corolario:}}

\(\chi(G) \geq 3 \iff \exists \text{ ciclo impar en } G\)\bigskip

\subsection*{Más sobre Greedy}
\begin{itemize}
\item Complejidad $Greedy = O(\sum d(n)) = O(n)$
\item Cota obvia $\chi(G) \leq \Delta + 1$

($Greedy$ nunca va a colorear con más de $\Delta + 1$ colores)
\end{itemize}

\begin{enumerate}
\item \(\chi(K_n) = n = (n-1) + 1 = \Delta + 1\)
\item \(\chi(C_{2r+1}) = 3 = 2+1 = \Delta + 1\)
\end{enumerate} \bigskip

Pregunta:

$\exists$ algun grafo conexo $\neq K_n, C_{2r+1}$ con $\chi(G) = \Delta +1 $?\bigskip

Respuesta: NO (teorema de Brooks, 1941)\bigskip

\newpage
\begin{center}
\subsection*{Teorema Baby Brooks}

$G$ conexo no regular $\implies \chi(G) \leq \Delta$ 
\end{center}

\underline{\textbf{Demostración:}}

sea $x \in V(G) : d(x) = \delta$\bigskip

corramos $BFS(x)$ (o $DFS(X)$)

esto da un orden de los vertices. El orden en que se fueron agregando a $BFS$\bigskip

corramos $Greedy$ con el orden \textbf{inverso} a ese

($\therefore x$ es el ultimo)

$x_1, x_2, ... , x_n$ con $x_n = x$\bigskip

tenemos $C(x_1) = 1$

recordemos $C(x_i) = min\left\{k : C(x_j) \neq k,  \forall j < 1 : x_{i,j} \in \Gamma(x)\right\}$\bigskip

En el orden $BFS$ todo vértice (salvo $x$) es agregado al arbol por un \textbf{vecino} que \textbf{ya esta} 

$\implies$ es anterior en el orden $BFS$

$\implies$ todo vertice $\neq x$ tiene un vecino \textbf{anterior} en el orden $BFS$

$\implies$ todo vertice $\neq x$ tiene un vecino \textbf{posterior} en el orden $BFS$ inverso\bigskip

$\therefore$ cuando voy  a colorear $x_i$, $x_i$ tiene a lo sumo $d(x_i) - 1$ vecinos anteriores en el orden $BFS$ inverso

$\therefore$ $Greedy$ elimina a lo sumo $d(x_i) - 1$ colores\bigskip

como $d(x_i) - 1 \leq \Delta -1$

$Greedy$ elimina a lo sumo $\Delta - 1$ colores al analizar $x_i$

$\implies$ va a colorear x con alguno de los colores $1, 2, ..., \Delta$ (eso si $i < n$)\bigskip

pero $i=n \implies x_n = x \implies d(x_n) = \delta < \Delta$

$\implies$ $Greedy$ elimina a lo sumo $\delta$ $\implies$ va a colorear a $x$ con algun color $\in\left\{1, 2, ..., \Delta\right\}$

\newpage

\section*{Clase del 19/3}
\subsection*{Ejemplo de Greedy "desastrozo"}
\(V =  \left \{ 1, 2, ..., 2r\right \},  r  \in  \mathbb{N}\)

 \( E =  \left \{ xy : x, y  \in V, x  \text{ impar, } y  \text{ par }  \ne x + 1\right \}\)

Ejemplo con $r = 4$

Ejemplo aqui

como no hay lados entre los vertices impares y no hay lados entre los lados pares, entonces $\chi(G) = 2$

\(C(x_i) =
\begin{cases}
1 & i \text{ par} \\
2 & i \text{ impar}
\end{cases} \)

corramos $Greedy$ en el orden \textit{natural} (1, 2, 3 , 4, 5, 6, 7, 8)

TABLA AQUI

$Greedy$ colorea en ese caso con $r$ colores y $r \ne 2$

\begin{center}
\subsection*{Teorema}

sea G un grafo que *ya tiene* un coloreo propio con $r$ colores (obtenidos de alguna forma, no necesariamente Greedy)

$\implies$ $Greedy$ en ese orden coloreara con a lo sumo $r$ colores

\end{center}

formalmente

sea $\pi : \left\{1, ..., r\right\} \rightarrow \left\{1, ..., r\right\}$ una biyección \\
sea $V_i = \left\{x \in V : C(x) = i\right\}$

si corremos $Greedy$ poniendo todos los vertices de $V_{\pi(1)}$ primero, luego todos los vertices de $V_{\pi(2)}$, etc. Terminando con los vertices de $V_{\pi(r)}$ al final, entonces tenemos $\leq r$ colores

\subsection*{Demostración}

sea $w_i = V_{\pi(1)} \cup V_{\pi(2)} \cup ... \cup V_{\pi(i)}$

hipótesis inductiva: $Greedy$ (en el orden del enunciado) colorea con $w_i$ con $\leq i$ colores

si probamos esto para todo $i$, como $w_r = V_r$ tendremos al teorema

\subsection*{Caso base}
$i = 1$

$w_i = V_{\pi(1)}$ vertices de color $\pi(1)$

como el coloreo original es propio y todos esos vertices tienen el mismo color, entonces no puede haber ningún lado entre ellos

$\therefore Greedy$ colorea a $w_1$ con 1 color


\subsection*{Paso inductivo}

asumamos por inducción que $Greedy$ colorea a $w_i$ con $\leq i$ colores y demostremos que Greedy coloreara a $w_{i+1}$ con $\leq i + 1$ colores

supongamos que no

$\implies$ debe haber un vertice $x \in w_{i+1}$ tal que $Greedy(x) = i + 2$

$\implies$ $Greedy$ no le pudo dar el color $i+1$

$\implies \exists y \in \Gamma(x)$ anterior a $x$ en el orden anterior tal que $Greedy(y) = i + 1$ 
 
como $x \in w_{i+1}$, $y$ es anterior a $x$ $\implies$ $y \in w_{i+1}$ también

$x, y \in w_{i+1}$

$\implies$ $x, y \notin w_i$ (porque $y$ tiene color $i+1$ y $x$ tiene color $i+2$)

$\implies x, y \in w_{i+1} - w_i$  ($w_{i+1} - w_i = V_{\pi(i+1)}$)

en resumen

$x,y \in V_{\pi(i+1)}, y \in \Gamma(x) \implies$ tienen el mismo color pero son vecinos $\implies$ ABSURDO

\subsection*{Corolario}

$\exists$ algún orden de los vertices tal que $Greedy$ colorea con exactamente $\chi(G)$ colores

tomemos un coloreo de G con $\chi(G)$ colores, reordenemos de acuerdo al teorema

$\implies$ $Greedy$ usara $\leq r$ colores $\implies$ $Greedy$ usara $\chi(G)$

\section*{Flujos en networks}

\subsection*{Grafo dirigido}

un grafo dirigido es un par $(V, E)$ con $E \subseteq V \times V$

notación: 

$\overrightarrow{xy} = (x, y)$

$\Gamma^{+}(x) = \left\{y \in V : \overrightarrow{xy} \in E\right\}$ 

$ \Gamma^{-}(x) = \left\{y \in V: \overrightarrow{yx} \in E\right\}$

un network es un triple $(V, E, C)$ con $(V, E)$ grafo dirigido y $C: E \rightarrow \mathbb{R} \geq 0$

\subsection*{Definición}

dada $f : E \rightarrow \mathbb{R} \geq 0$, $x \in V$ denotamos

$out_{f}(x) = \sum_{y \in \Gamma^{+}(x)}^{} f(x, y)$

$in_{f}(x) = \sum_{y \in \Gamma^{-}(x)}^{} f(x, y)$

\subsection*{Definición}

dado un network $N = (V, E, C)$ y $s,t \in V$

definimos un flujo en $N$ de $s$ a $t$ como una función

$f: E \rightarrow \mathbb{R}$ tal que

\begin{enumerate}
\item $0 \leq f(x, y) \leq C(x,y)$  $\forall(x, y) \in E$
\item $out_{f}(x) = in_{f}(x)$ $\forall x \ne s, t$    'ley de conservación'
\item $in_{f}(s) = 0$      $s$ es productor, alternativamente $in_{f}(s) \leq out_{f}(s)$
\item $out_{f}(t) = 0$    $t$ es consumidor, alternativamente $out_{f}(s) \leq in_{f}(s)$
\end{enumerate}

\subsection*{Propiedad: cantidad obvia}

sea $f$ flujo de $s$ a $t$ $\implies$ $out_{f}(s) = in_{f}(s)$

\subsection*{Definición:}

el valor de $f$ es $\gamma(f) = out_f(s)$       $(=in_f(t))$

\subsection*{Max flow problem - Problema del flujo maximal}

dado un network $N$ y $s, t \in V$, hallar un flujo 'maximal' de $s$ a $t$. Es decir, $f$ flujo de $s$ a $t$ tal que 

$\gamma(g) \leq \gamma(f)$ $\forall$ flujo $g$ de $s$ a $t$


\subsection*{Definición}

un camino dirigido en un network (o cualquier grafo) desde un vertice $x$ a un vertice $y$ (entre $x$ e $y$) es una sucesión de vertices $x_0,x_1, .. x_r$ tales que $x_0 = x$ y $x_r = y$

$\overrightarrow{x_{i}x_{i+1}} \in E $    $\forall 0 \leq i < r$


\subsection*{Definición}

un lado $\overrightarrow{xy}$ se dice saturado (relativo a un flujo $f$) si $f(x) = C(x)$

\subsection*{Algoritmo 'Greedy' para flujo maximal}

\begin{enumerate}
\item $f = 0 \rightarrow f(x,y) = 0$    $\forall xy$
\item
\texttt{while (1) \{}

\hspace{1em} \texttt{if } ($\exists$ un camino dirigido de $s$ a $t$ sin ningun lado saturado)

\hspace{3em} sea $x_0 = s, x_1, ..., x_r = t$ un camino

\hspace{3em} $\varepsilon = min \left\{C(\overrightarrow{x_ix_{i+1}}) - f(\overrightarrow{x_ix_{i+1}}) : 0 \leq i < r\right\}$

\hspace{3em} mandar $\epsilon$ unidades de flujo por el camino

\hspace{3em} $f(\overrightarrow{xy})$ \texttt{=} $f(\overrightarrow{xy}) + \varepsilon$

\hspace{1em} \texttt{else}

\hspace{3em} \texttt{return (f);}

\texttt{\}}
\end{enumerate}

\subsection*{Complejidad de Greedy}

en cada iteración del \texttt{while} se satura al menos un lado $\implies$ hay $O(n)$ iteraciones

complejidad de cada iteración:

hay que encontrar un camino dirigido 'no saturado' $\rightarrow$ usamos $BFS/DFS$ $\implies O(n)$
ademas, sumamos $O(n)$ de calcular $\varepsilon$ y aumentar $\varepsilon$

complejidad total = $O(n^2)$ es polinomial

EJEMPLO DE GREEDY ACA


Ahora vamos a 'alterar' a Greedy permitiendo *otros* caminos, 'no dirigidos'. veremos el Algoritmo de Ford-Fulkerson, que es esencialmente Greedy con esos caminos 'augmenting paths'

\section*{Motivación para caminos aumentantes}

EJEMPLO ACA

\subsection*{Definición}
un camino 'aumentante' de $x$ a $y$ es una sucesión de vertices $x_0 = x,  x_1, ..., x_r = y$ tal que $\forall 0 \leq i < r$ pasa una de las 2 cosas siguientes

\begin{enumerate}
\item \(\overrightarrow{x_ix_{i+1}} \in E \land f(\overrightarrow{x_ix_{i+1}}) < C(\overrightarrow{x_ix_{i+1}})\)
\item \(\overrightarrow{x_{i+1}x_i} \in E \land f(\overrightarrow{x_{i+1}x_i}) > 0\)
\end{enumerate}

con casos 1) 'Forward'  y 2) 'Backward'

\subsection*{Definición}

un corte es un subconjunto $S \subseteq V$ tal que $s \in S, t \notin S$

\subsection*{Notación}
sea $f: E \rightarrow \mathbb{R}$

vamos a extender a $f$ de pares de vertices a pares de conjuntos de vertices si $A,B \subseteq V$

\(
f(A,B) = \sum_{\substack{x \in A \\ y \in B \\ \overrightarrow{xy} \in E}} f(x, y)
\)

\begin{center}
\section*{Teorema}
sea $f$ flujo y $S$ corte

$\gamma(f) = f(S,\overline{S}) - f(\overline{S}, S)$
\end{center}

\subsection*{Demostración}

calculamos $\sum_{x \in S}out_f(x) - in_f(x)$ \\

\(= (out_f(s) - in_f(s)) + \sum_{\substack{x \in S \\ x \ne s}}out_f(x) - in_f(x)\) \hspace{12em}(definición de $\sum$ para $x=s$)\\ 

\(= (out_f(s) - 0) + 0\) \hspace{16em}(definicion de $in_f(s)$ y $out_f(x) = in_f(x) \forall x \ne s $)\\

\(= out_f(s)\) \\

\(= \gamma(f)\) \\

\(\implies \gamma(f) = \sum_{x \in S}(out_f(x)-in_f(x))\) \\

\(= \sum_{x \in S}out_f(x) - \sum_{x \in S}in_f(x)\) \hspace{26em} (def de $\sum$) \\

\(= \sum_{x \in S}\sum_{y \in \Gamma^{+}(x)}f(x, y) - \sum_{x \in S}\sum_{y \in \Gamma^{-}(x)}f(y,x)\) \hspace{12em} (def de $out_f(x)$ y $in_f(x)$)\\

\(= \sum_{\substack{x \in S \\ y \in V \\ \overrightarrow{xy} \in E}}f(x,y) - \sum_{\substack{x \in S \\ y \in V \\ \overrightarrow{xy} \in E}}f(y,x)\)\\

pero, observemos por un momento que si $A\cap B = \emptyset$, entonces\\

\(f(A \cup B, C) = f(A,C) + f(B, C)\)\\

\(f(C, A \cup B) = f(C, A) + f(C, B)\)\\

Entonces, continuamos con la demostración\\

\(= \sum_{\substack{x \in S \\ y \in V \\ \overrightarrow{xy} \in E}}f(x,y) - \sum_{\substack{x \in S \\ y \in V \\ \overrightarrow{xy} \in E}}f(y,x) = f(S, S \cup \overline{S}) - f(S \cup \overline{S}, S)\) \\

\(= f(S,S) + f(S, \overline{S}) - f(S, S) - f(\overline{S}, S)\)\\

\(= f(S, \overline{S}) - f(\overline{S}, S)\)

\section*{Algoritmo de Ford-Fulkerson}
\begin{enumerate}
\item \(f = 0\) (o con cualquier flujo)
\item \texttt{while} ($\exists$ $f$-camino aumentante entre $s$ y $t$)

\hspace{3em} tomar $s = x_0, x1, ..., t = x_r$ un camino aumentante

\hspace{3em} 
\(\varepsilon_i =
\begin{cases}
C(\overrightarrow{x_{i}x_{i+1}}) - f(\overrightarrow{x_{i}x_{i+1}}) & \text{ si } \overrightarrow{x_{i}x_{i+1}} \in E \\
f(\overrightarrow{x_{i+1}x_i}) & \text{ si } \overrightarrow{x_{i+1}x_i} \in E
\end{cases} 
\)

\hspace{3em} \(\varepsilon = min(\varepsilon_i)\)

\hspace{3em} aumentar $f$ en $\varepsilon$ a lo largo del camino *

\texttt{end while}

\item \texttt{return \((f)\)}
\end{enumerate}

*nota: 'aumentar $f$ en $\varepsilon$' : 

$f_{nuevo}(\overrightarrow{x_ix_{i+1}}) = f_{viejo}(\overrightarrow{x_ix_{i+1}}) + \varepsilon$ caso 'forward'

$f_{nuevo}(\overrightarrow{x_{i+1}x_{i}}) = f_{viejo}(\overrightarrow{x_{i+1}x_{i}}) + \varepsilon$ caso 'backward'

\begin{center}
\section*{Teorema}
si aumentamos $f$ en $\varepsilon$, lo que queda es flujo
\end{center}

\subsection*{Demostración}

si $\overrightarrow{xy}$ no forma parte del camino $f$, queda igual

si tengo un lado forward $\overrightarrow{x_{i}x_{i+1}}$, llamaremos $f^*$ a lo que obtenemos luego de aumentar\\

\(f^*(\overrightarrow{x_{i}x_{i+1}}) = f(\overrightarrow{x_{i}x_{i+1}}) + \varepsilon\)

\(f^*(\overrightarrow{x_{i}x_{i+1}}) \geq 0 + \varepsilon\)

\(f^*(\overrightarrow{x_{i}x_{i+1}}) \geq  \varepsilon\ > 0\)\\



y\\

\(f^*(\overrightarrow{x_{i}x_{i+1}}) = f(\overrightarrow{x_{i}x_{i+1}}) + \varepsilon\)

\(f^*(\overrightarrow{x_{i}x_{i+1}}) \leq f(\overrightarrow{x_{i}x_{i+1}}) + \varepsilon_i \)

\(f^*(\overrightarrow{x_{i}x_{i+1}}) \leq  f(\overrightarrow{x_{i}x_{i+1}}) +  C(\overrightarrow{x_{i}x_{i+1}}) - f(\overrightarrow{x_{i}x_{i+1}})\)\hspace{18em}(def de $\varepsilon_i$)

\(f^*(\overrightarrow{x_{i}x_{i+1}}) \leq C(\overrightarrow{x_{i}x_{i+1}}) \)\\

\(\implies 0 < f^*(\overrightarrow{x_{i+1}x_{i}}) \leq C(\overrightarrow{x_{i+1}x_{i}} \)\\

caso backward

\(f^*(\overrightarrow{x_{i+1}x_{i}}) = f(\overrightarrow{x_{i+1}x_{i}}) - \varepsilon\)

\(f^*(\overrightarrow{x_{i+1}x_{i}}) < f(\overrightarrow{x_{i+1}x_{i}}\)

\(f^*(\overrightarrow{x_{i+1}x_{i}}) < C(\overrightarrow{x_{i+1}x_{i}}\)\\

y \\

\(f^*(\overrightarrow{x_{i+1}x_{i}}) = f(\overrightarrow{x_{i+1}x_{i}}) - \varepsilon\)

\(f^*(\overrightarrow{x_{i+1}x_{i}}) \geq f(\overrightarrow{x_{i+1}x_{i}}) - \varepsilon\)

\(f^*(\overrightarrow{x_{i+1}x_{i}}) \geq 0\)\\

\(\implies 0 \leq f^*(\overrightarrow{x_{i+1}x_{i}}) <C(\overrightarrow{x_{i+1}x_{i}} \)\\

conclusión:

en cualquiera de los 2 casos

\(0 \leq f^*(\overrightarrow{xy}) \leq C(\overrightarrow{xy}) \)\\

queda ahora determinar que $in_{f^*}(x) = out_{f^*}(x)$ $\forall x \ne s, t$
\newpage

\section*{Clase del 21/3}
Lorem ipsum dolor sit amet, consectetur adipiscing elit, sed do eiusmod tempor incididunt ut labore et dolore magna aliqua. Ut enim ad minim veniam, quis nostrud exercitation ullamco laboris nisi ut aliquip ex ea commodo consequat. Duis aute irure dolor in reprehenderit in voluptate velit esse cillum dolore eu fugiat nulla pariatur. Excepteur sint occaecat cupidatat non proident, sunt in culpa qui officia deserunt mollit anim id est laborum.

\end{document}
   
